% !TEX root = ../thesis-sample.tex
\appendix
\chapter{Apéndice capítulo 6: Patrones}


\section{Patrón composite}

\subsection{Clase Tarea}
\begin{lstlisting}[language = Java  , firstnumber = last , escapeinside={(*@}{@*)}]
public abstract class Tarea{
public abstract void operacion();
public void adicionar(Componente componente){}
public void remover(Componente componente){}
public Componente getHijo(int a){
  return null;
}
}
\end{lstlisting}

\subsection{Clase Subtarea}
\begin{lstlisting}[language = Java  , firstnumber = last , escapeinside={(*@}{@*)}]
import java.util.List;
import java.util.ArrayList;
public class Subtarea extends Tarea{
private List<Tarea> hijo=new ArrayList<Tarea>();
public void operacion(){
  //para todo g en hijo
  for(Tarea g:hijo){
     g.operacion();
  }
}
public void adicionar(Componente componente){
  hijo.add(componente);
}
public void remover(Componente componente){
  hijo.remove(componente);
}
public Componente getHijo(int a){
  return hijo.get(a);
}
}
\end{lstlisting}

\subsection{Clase Tarea trabajo}
\begin{lstlisting}[language = Java  , firstnumber = last , escapeinside={(*@}{@*)}]
public class TareaTrabajo extends Tarea{
private String descripcionTrabajo;
}
\end{lstlisting}

\subsection{Clase Tarea consulta}
\begin{lstlisting}[language = Java  , firstnumber = last , escapeinside={(*@}{@*)}]
public class TareaConsulta extends Tarea{
private String descripcionConsulta;
private boolean requiereConexion;
public void operacion(){}
}
\end{lstlisting}

\subsection{Clase Tarea lectura}
\begin{lstlisting}[language = Java  , firstnumber = last , escapeinside={(*@}{@*)}]
public class TareaLectura extends Tarea{
private int numeroPaginas;
private String nombreLectura;
public void operacion(){null}
}
\end{lstlisting}

\section{Patrón agrupador}

\subsection{Clase horario}

\begin{lstlisting}[language = Java  , firstnumber = last , escapeinside={(*@}{@*)}]
import java.util.List;
import java.util.ArrayList;
public abstract class Horario{
	protected static Horario localizador;
	protected List<Horario> argupador=new ArrayList<Horario>();
	public static void main(String[] args){
		Horario A =  new Franja();
		A.abrirGrupo();
		A.agrupar();
		A.manejarAgrupacion();
		Horario B =  new AgrupadorB();
		B.agrupar();
		B.manejarAgrupacion();
	}
	public void abrirGrupo(){
		if(localizador==null){
   			localizador=this;
   			argupador=new ArrayList<Horario>();
   		}
	}
	public void cerrarGrupo(){
		localizador = null;
	}
	public void agrupar(){
		if(localizador!=null){
   			(argupador = localizador.argupador).add(this);
		}
	}
	public void desagrupar(){
		if(localizador!=null){
   			localizador.argupador.remove(this);
		}
	}
	public void manejarAgrupacion(){
		for(Horario a:argupador){
 	    	a.accion();
		}
	}
	protected abstract void accion();
}
\end{lstlisting}

\subsection{Clase franja}

\begin{lstlisting}[language = Java  , firstnumber = last , escapeinside={(*@}{@*)}]
import static com.componentes.diseno.lmc.marcosDeReferencia.computacion.Computacion.*;
import static com.componentes.diseno.lmc.marcosDeReferencia.emoticons.Emoticons.*;
public class Franja extends Horario{
	private String nombreFranja;
	private int tiempoInicio;
	private int tiempoFinal;
	private int horas;
	private int disponibilidad;
	private []int dias;
	protected void accion(){
		System.out.println(SERIOUS);
	}
}
\end{lstlisting}
\section{Patrón fábrica abstracta}

\section{Patrón estrategia}
