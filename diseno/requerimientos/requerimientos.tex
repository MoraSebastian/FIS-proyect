
\chapter{Requerimientos}
\section{Introducción}
Como cualquier proyecto de cualquier indole, un punto fundamental, consiste en conocer cual es la necesidad o el problema que se desea resolver, lo cual conlleva a saber qué es lo que se debe realizar, para solventar tales necesidades. Esta es la idea básica de los requerimientos, definir cuales son los aspectos que se deben realizar para complacer las necesidades de un cliente.
\newpage

\section{Requerimientos del Cliente}
Se entienden como el qué es lo que el cliente esperará encontrar, cuando interactúe con la aplicación. Bajo la anterior premisa, se procede a definir los siguientes requerimientos de cliente:

\begin{enumerate}
	\item Mostrar todas las tareas pendientes.
	\item Mostrar las tareas pendientes por categoría.
	\item Mostrar las tareas pendientes por su tipo.
	\item Mostrar las tareas pendientes para una fecha.
	\item Mostrar las tareas pendientes por dificultad.
	\item Mostrar los horarios asignados para las tareas pendientes.
	\item Alertar de la próxima entrega de una tarea.
	\item Sugerir horarios de clase del usuario.
	\item Sugerir cuánto tiempo podría tomar una tarea.
	\item Sugerir tiempos de pausas activas durante la realización de una tarea.
	\item Advertir si se debe sacrificar algún espacio destinado a descanso.
	\item Advertir cuando no se alcanzará a terminar una tarea para una fecha especificada.
\end{enumerate}



\begin{table}[htb]
\centering
\begin{tabular}{|l|c|p{8cm}|}
\hline
RF- & \multicolumn {2}{p{10cm}|} {Nombre del requisito funcional una vaina asi bien larga carajo pero que no se pase jajajaj }    \\
\hline
Descripcion & \multicolumn {2}{p{10cm}|} {la descripcion}\\
\hline
Precondición & \multicolumn {2}{p{10cm}|} {la precondicion}\\
\cline{2-3}
Secuencia & Paso & Acción \\
\cline{2-3}
& 1 & relatar acción \\
\cline{2-3}
& 2 & relatar acción 2 \\
\hline
Postcondición & \multicolumn {2}{p{10cm}|} {la postcondición} \\
\hline
Excepciones & Paso & Acción \\
\cline{2-3}
& 1 & relatar acción haber probando si se pasa bien o se pasa mal espero que  sea bien jajajaja  \\
\cline{2-3}
& 2 & relatar acción 2 \\
\cline{2-3}
Rendimiento & Paso & Cota de tiempo \\
\cline{2-3}
& 1 & cota de tiempo 1 \\
\cline{2-3}
& 2 & cota de tiempo 2 \\
\hline
Importancia & \multicolumn {2}{p{10cm}|} {la importancia}    \\
\hline
Urgencia & \multicolumn {2}{p{10cm}|} {la urgencia}    \\
\hline
\end{tabular}
\end{table}