
\chapter{Requerimientos}
\section{Introducción}
Para cualquier proyecto de software, es un punto fundamental conocer cuál es la necesidad y el problema que el cliente desea resolver. Para tener una visión holística del problema, se hace necesario definir los requerimientos que satisfagan al cliente y resuelvan el problema.

\section{Requerimientos del Cliente}
Se entiende como lo que el cliente espera encontrar cuando interactúe con la aplicación. Bajo la anterior premisa, se definieron los siguientes requerimientos:
\begin{enumerate}
	\item Añadir una tarea.
	\item Añadir subtareas para una tarea.
	\item Añadir un horario universitario.
	\item Añadir un horario de descanso (dormir).
	\item Añadir un horario de transporte.
	\item Añadir una tarea a una materia.
	\item Mostrar todas las tareas pendientes.
	\item Mostrar las tareas pendientes por materia.
	\item Mostrar las tareas pendientes por tipo.
	\item Mostrar las tareas pendientes para una fecha.
	\item Mostrar las tareas pendientes por dificultad.
	\item Mostrar el horario general del usuario.
	\item Mostrar los horarios asignados para las tareas pendientes.
	\item Modificar horario.
	\item Modificar tarea.
	\item Sugerir horarios para realizar tareas.
	\item Sugerir cuánto tiempo podría tomar una tarea.
	\item Sugerir tiempos de pausas activas durante la realización de una tarea.
	\item Alertar de la próxima entrega de una tarea.
	\item Advertir si se debe sacrificar algún espacio de descanso.

\end{enumerate}

\section{Casos de uso}
Los casos de uso describen la interacción del usuario con las diversas funcionalidades planteadas, permitiendo obtener una forma de comunicar los requerimientos de tal forma que sea entendida tanto por usuario como por desarrolladores. Como se podrá observar a continuación, serán cuatro diagramas de caso de uso los que se presentan, donde el motivo por el cual los casos de uso comparten diagrama, es porque se considera que existe cierta relación entre ellos.



\begin{figure}[H]
	\centering
	\includegraphics[width=0.7\linewidth]{diseno/requerimientos/imagenes/casouso1}
	\caption{Primer diagrama de caso de uso}
	\label{fig:gantt}
\end{figure}
\begin{figure}[H]
	\centering
	\includegraphics[width=0.7\linewidth]{diseno/requerimientos/imagenes/casouso2}
	\caption{Segundo diagrama de caso de uso}
	\label{fig:gantt}
\end{figure}
\begin{figure}[H]
	\centering
	\includegraphics[width=0.7\linewidth]{diseno/requerimientos/imagenes/casouso3}
	\caption{Tercero diagrama de caso de uso}
	\label{fig:gantt}
\end{figure}
\begin{figure}[H]
	\centering
	\includegraphics[width=0.7\linewidth]{diseno/requerimientos/imagenes/casouso4}
	\caption{Cuarto diagrama de caso de uso}
	\label{fig:gantt}
\end{figure}

Los diagramas a continuación representan gran importancia complemetando la definición de los requerimientos.

\section{Diagramas de secuencia}
Los diagramas de secuencia permiten obervar la realización del caso de uso, responden el como se va a hacer el requerimiento. Los siguientes son los diagramas de secuencia de 4 casos de uso que se consideran de mayor importancia.

\begin{figure}[H]
	\centering
	\includegraphics[width=0.7\linewidth]{diseno/requerimientos/imagenes/secuenciaE01.png} 
	\caption{Diagrama de secuencia caso de uso 01.}
	\label{fig:gantt}
\end{figure}

\begin{figure}[H]
	\centering
	\includegraphics[width=0.7\linewidth]{diseno/requerimientos/imagenes/secuenciaE02.png} 
	\caption{Diagrama de secuencia caso de uso 02.}
	\label{fig:gantt}
\end{figure}

\begin{figure}[H]
	\centering
	\includegraphics[width=0.7\linewidth]{diseno/requerimientos/imagenes/secuenciaE03.png} 
	\caption{Diagrama de secuencia caso de uso 03.}
	\label{fig:gantt}
\end{figure}

\begin{figure}[H]
	\centering
	\includegraphics[width=0.7\linewidth]{diseno/requerimientos/imagenes/secuenciaE04.png} 
	\caption{Diagrama de secuencia caso de uso 05.}
	\label{fig:gantt}
\end{figure}

\begin{figure}[H]
	\centering
	\includegraphics[width=0.7\linewidth]{diseno/requerimientos/imagenes/secuencia1}
	\caption{Diagrama de secuencia caso de uso 14.}
	\label{fig:gantt}
\end{figure}
\begin{figure}[H]
	\centering
	\includegraphics[width=0.7\linewidth]{diseno/requerimientos/imagenes/secuencia2}
	\caption{Diagrama de secuencia caso de uso 15.}
	\label{fig:gantt}
\end{figure}
\begin{figure}[H]
	\centering
	\includegraphics[width=0.7\linewidth]{diseno/requerimientos/imagenes/secuencia3}
	\caption{Diagrama de secuencia caso de uso 16.}
	\label{fig:gantt}
\end{figure}
\begin{figure}[H]
	\centering
	\includegraphics[width=0.7\linewidth]{diseno/requerimientos/imagenes/secuencia4}
	\caption{Diagrama de secuencia caso de uso 19.}
	\label{fig:gantt}
\end{figure}

\section{Diagramas de comunicación}
Igualmente relacionados con los diagramas anteriores, principalmente con el diagrama de secuencia. Su función como su nombre lo indica, consiste en detallar en como se comunican los objetos que solucionan el requerimiento.

Se presentan los diagramas correspondientes a los ya expuestos diagramas de secuencia:

\begin{figure}[H]
	\centering
	\includegraphics[width=0.7\linewidth]{diseno/requerimientos/imagenes/comunicacionE01.png} 
	\caption{Diagrama de comunicación caso de uso 01.}
	\label{fig:gantt}
\end{figure}

\begin{figure}[H]
	\centering
	\includegraphics[width=0.7\linewidth]{diseno/requerimientos/imagenes/comunicacionE02.png} 
	\caption{Diagrama de comunicación caso de uso 02.}
	\label{fig:gantt}
\end{figure}

\begin{figure}[H]
	\centering
	\includegraphics[width=0.7\linewidth]{diseno/requerimientos/imagenes/comunicacionE03.png} 
	\caption{Diagrama de comunicación caso de uso 03.}
	\label{fig:gantt}
\end{figure}

\begin{figure}[H]
	\centering
	\includegraphics[width=0.7\linewidth]{diseno/requerimientos/imagenes/comunicacionE05.png} 
	\caption{Diagrama de comunicación caso de uso 05.}
	\label{fig:gantt}
\end{figure}

\begin{figure}[H]
	\centering
	\includegraphics[width=0.7\linewidth]{diseno/requerimientos/imagenes/comunicacion1}
	\caption{Diagrama de comunicación caso de uso 14.}
	\label{fig:gantt}
\end{figure}
\begin{figure}[H]
	\centering
	\includegraphics[width=0.7\linewidth]{diseno/requerimientos/imagenes/comunicacion2}
	\caption{Diagrama de comunicación caso de uso 15.}
	\label{fig:gantt}
\end{figure}
\begin{figure}[H]
	\centering
	\includegraphics[width=0.7\linewidth]{diseno/requerimientos/imagenes/comunicacion3}
	\caption{Diagrama de comunicación caso de uso 16.}
	\label{fig:gantt}
\end{figure}
\begin{figure}[H]
	\centering
	\includegraphics[width=0.7\linewidth]{diseno/requerimientos/imagenes/comunicacion4}
	\caption{Diagrama de comunicación caso de uso 19.}
	\label{fig:gantt}
\end{figure}

Las siguientes tablas, son la especificación de los requerimientos que se considera que tienen un nivel de importancia alta.

 % 1.
\begin{table}[htb]
\centering
\begin{tabular}{|l|c|p{8cm}|}
\hline
RF-01 & \multicolumn {2}{p{10cm}|} {Añadir una tarea }    \\
\hline
Descripción & \multicolumn {2}{p{10cm}|} {El usuario añade una tarea pendiente por desarrollar. }\\
\hline
Precondición & \multicolumn {2}{p{10cm}|} {El usuario debe tener un horario}\\
\cline{2-3}
Secuencia & Paso & Acción \\
\cline{2-3}
& 1 & El usuario selecciona la opción de crear tarea.\\
\cline{2-3}
& 2 & El usuario proporciona la información requerida (nombre de la tarea, tipo, materia a la que pertenece) \\
& 3 & El usuario verifica la información registrada. \\
& 4 & El usuario hace selecciona el botón aceptar. \\
\hline
Postcondición & \multicolumn {2}{p{10cm}|} {El sistema muestra la tarea recién asignada con sus especificaciones y su recomendación de tiempo de realización y de horario } \\
\hline
Excepciones & Paso & Acción \\
\cline{2-3}
& 4 & Se  añade una tarea que requiere urgencia (Imprevisto). El usuario elige que horario sacrificará para realizar la tarea. \\
\cline{2-3}
& 4 & Se añade una tarea que es imposible de realizar debido al tiempo u horario. Es necesario modificar los tiempos u horarios en los que se realizará la tarea o elegir si sacrificar una frnaja de horario. \\
\hline
Rendimiento & Paso & Cota de tiempo \\
\cline{2-3}
& 1 & 1 segundo \\
\cline{2-3}
& 2 & 40 segundos \\
\cline{2-3}
& 3 & 5  segundos \\
\cline{2-3}
& 4 & 1 segundo \\
\hline
Importancia & \multicolumn {2}{p{10cm}|} {Muy importante}    \\
\hline
Urgencia & \multicolumn {2}{p{10cm}|} {urgente}    \\
\hline

\end{tabular}
\end{table}

% 2.
\begin{table}[htb]
\centering
\begin{tabular}{|l|c|p{8cm}|}
\hline
RF-02 & \multicolumn {2}{p{10cm}|} {Añadir subtareas para una tarea. }    \\
\hline
Descripcion & \multicolumn {2}{p{10cm}|} {Se añade una subtarea a una tarea.}\\
\hline
Precondición & \multicolumn {2}{p{10cm}|} {Debe existir alguna tarea pendiente.}\\
\cline{2-3}
Secuencia & Paso & Acción \\
\cline{2-3}
& 1 & El usuario selecciona la tarea a la que desea añadirle una subtarea. \\
\cline{2-3}
& 2 & Seleccionar la opción de añadir subtarea. \\
\cline{2-3}
& 3 & Se añade la subtarea como una tarea (RF-01) \\
\cline{2-3}
& 4 & El usuario verifica la información. \\
\cline{2-3}
& 5 & El pulsa la opción de aceptar. \\
\hline
Postcondición & \multicolumn {2}{p{10cm}|} {El sistema añadirá la subtarea a la tarea, mostrará sus especificaciones y recomendación de tiempo de realización y de horario} \\
\hline
Excepciones & Paso & Acción \\
\cline{2-3}
& 1 & No existe una tarea para añadirle una subtarea.  \\
\cline{2-3}
& 3 & La subtarea es de carácter urgente.
 \\
\cline{2-3}
& 4 & Se añade una subtarea y esta hace que la tarea sea imposible de terminar debido al tiempo u horario. \\

\hline
Rendimiento & Paso & Cota de tiempo \\
\cline{2-3}
& 1 & 5 segundos \\
\cline{2-3}
& 2 & 1 segundo \\
\cline{2-3}
& 3 & 40 segundos \\
\cline{2-3}
& 4 & 5 segundos  \\
\cline{2-3}
& 5 & 1 segundo \\
\hline
Importancia & \multicolumn {2}{p{10cm}|} {Importante}    \\
\hline
Urgencia & \multicolumn {2}{p{10cm}|} {No urgente}    \\
\hline
Comentarios & No. & Descripción \\
\cline{2-3}
& 1 & Añadir una subtarea es lo mismo que añadir una tarea, la deferencia es que está anidada dentro de una tarea general. \\
\hline
\end{tabular}
\end{table}

% 3.
\begin{table}[htb]
\centering
\begin{tabular}{|l|c|p{8cm}|}
\hline
RF-03 & \multicolumn {2}{p{10cm}|} {Añadir un horario universitario }    \\
\hline
Descripción & \multicolumn {2}{p{10cm}|} {Se crea un horario con materias de la universidad.}\\
\hline
Precondición & \multicolumn {2}{p{10cm}|} {Ser un usuario registrado.}\\
\cline{2-3}
Secuencia & Paso & Acción \\
\cline{2-3}
& 1 & Seleccionar la opción de crear horario. \\
\cline{2-3}
& 2 & Escribir el nombre de cada materia y su respectiva hora de inicio y fin y los días en que se repite. \\
\cline{2-3}
& 3 & El usuario añade la materia y repite el proceso cuantas veces sea necesario. \\
\cline{2-3}
& 4 & Pulsar en el botón de aceptar. \\
\hline
Postcondición & \multicolumn {2}{p{10cm}|} {El sistema guardará el horario asignado para el usuario. } \\
\hline
Excepciones & Paso & Acción \\
\cline{2-3}
& 3 & El horario de la universidad llena totalmente los espacios disponibles.
 \\
\cline{2-3}
& 3 & No hay espacios disponibles para añadir más materias al horario.
 \\
\hline
Rendimiento & Paso & Cota de tiempo \\
\cline{2-3}
& 1 & 1 segundo \\
\cline{2-3}
& 2 & 40 segundos \\
\cline{2-3}
& 3 & 2 minutos \\
\cline{2-3}
& 4 & 1 segundo \\
\hline
Importancia & \multicolumn {2}{p{10cm}|} {Muy importante}    \\
\hline
Urgencia & \multicolumn {2}{p{10cm}|} {Urgente}    \\
\hline
\end{tabular}
\end{table}

% 4.
\begin{table}[htb]
\centering
\begin{tabular}{|l|c|p{8cm}|}
\hline
RF-04 & \multicolumn {2}{p{10cm}|} {Mostrar todas las tareas pendientes. }    \\
\hline
Descripcion & \multicolumn {2}{p{10cm}|} {Se muestra la lista de tareas pendientes.}\\
\hline
Precondición & \multicolumn {2}{p{10cm}|} {Debe existir al menos una tarea pendiente.}\\
\cline{2-3}
Secuencia & Paso & Acción \\
\cline{2-3}
& 1 & Seleccionar la opción de ver las tareas pendientes. \\
\hline
Postcondición & \multicolumn {2}{p{10cm}|} {El sistema mostrará todas las tareas pendientes} \\
\hline
Excepciones & Paso & Acción \\
\cline{2-3}
& 1 & No hay tareas pendientes para mostrar.  \\
\hline
Rendimiento & Paso & Cota de tiempo \\
\cline{2-3}
& 1 & 1 segundo \\
\hline
Importancia & \multicolumn {2}{p{10cm}|} {Vital}    \\
\hline
Urgencia & \multicolumn {2}{p{10cm}|} {Urgente?}    \\
\hline
\end{tabular}
\end{table}

% 5.
\begin{table}[htb]
\centering
\begin{tabular}{|l|c|p{8cm}|}
\hline
RF-05 & \multicolumn {2}{p{10cm}|} {Mostrar el horario general del usuario. }    \\
\hline
Descripción & \multicolumn {2}{p{10cm}|} {Se muestra el horario completo del estudiante.}\\
\hline
Precondición & \multicolumn {2}{p{10cm}|} {El usuario debe haber creado un horario antes.}\\
\cline{2-3}
Secuencia & Paso & Acción \\
\cline{2-3}
& 1 & Seleccionar la opción de mostrar el horario. \\

\hline
Postcondición & \multicolumn {2}{p{10cm}|} {El sistema mostrará el horario con las materias, los descansos, los horario de transporte y las tareas pendientes} \\
\hline
Excepciones & Paso & Acción \\
\cline{2-3}
& 1 & No hay un horario para presentar.  \\

\hline
Rendimiento & Paso & Cota de tiempo \\
\cline{2-3}
& 1 & 1 segundo \\
\hline
Importancia & \multicolumn {2}{p{10cm}|} {Importante}    \\
\hline
Urgencia & \multicolumn {2}{p{10cm}|} {Puede esperar}    \\
\hline
\end{tabular}
\end{table}


% 14.
\begin{table}[htb]
\centering
\begin{tabular}{|l|c|p{8cm}|}
\hline
RF-14 & \multicolumn {2}{p{10cm}|} {Modificar horarios}    \\
\hline
Descripcion & \multicolumn {2}{p{10cm}|} {Se selecciona y modifica una franja del horario.}\\
\hline
Precondición & \multicolumn {2}{p{10cm}|} {El horario que lo que se desea modificar debe estar asignado.}\\
\cline{2-3}
Secuencia & Paso & Acción \\
\cline{2-3}
& 1 & Se selecciona la opción de modificar horario. \\
\cline{2-3}
& 2 & Se selecciona el horario de la tarea que se desea cambiar. \\
\cline{2-3}
& 3 & Se selecciona la nueva franja de horario en la que se acomodara la tarea. \\
\cline{2-3}
& 4 & El cambio de horario se ha realizado. \\
\hline
Postcondición & \multicolumn {2}{p{10cm}|} {El horario es modificado y el sistema puede ofrecer sugerencia de tiempo de realización, o incluso si se debe sacrificar algún espacio de descanso.} \\
\hline
Excepciones & Paso & Acción \\
\cline{2-3}
& 1 & No hay ningún horario para seleccionar, en este caso el caso de uso acaba.  \\
\cline{2-3}
& 3 & No existe ninguna franja disponible para cambiar, en este caso el caso de uso acaba.\\
\hline
Rendimiento & Paso & Cota de tiempo \\
\cline{2-3}
& 1 & 1 segundo \\
\cline{2-3}
& 2 & 1 segundo \\
\cline{2-3}
& 3 & 10 segundos \\
\cline{2-3}
& 4 & 1 segundo \\
\hline
Importancia & \multicolumn {2}{p{10cm}|} {Importante}    \\
\hline
Urgencia & \multicolumn {2}{p{10cm}|} {Hay presión}    \\
\hline


\end{tabular}
\end{table}

% 15.
\begin{table}[htb]
\centering
\begin{tabular}{|l|c|p{8cm}|}
\hline
RF-15 & \multicolumn {2}{p{10cm}|} {Modificar Tareas. }    \\
\hline
Descripcion & \multicolumn {2}{p{10cm}|} {Se permite modificar los diferentes campos de una tarea.}\\
\hline
Precondición & \multicolumn {2}{p{10cm}|} {Debe existir alguna tarea para modificar.}\\
\cline{2-3}
Secuencia & Paso & Acción \\
\cline{2-3}
& 1 & Se solicita modificar una tarea. \\
\cline{2-3}
& 2 & El usuario selecciona la tarea que desea modificar. \\
\cline{2-3}
& 3 & El usuario selecciona el campo de la tarea que desea modificar. \\
\cline{2-3}
& 4 & Se modifica el campo de la tarea. \\
\cline{2-3}
& 5 & Se permite elegir realizar otro cambio o terminar. \\
\hline
Postcondición & \multicolumn {2}{p{10cm}|} {La tarea tiene un campo modificado, según el tipo podría haber una sugerencia, como en el caso de dificultad, o cambio de horario para realizarse.} \\
\hline
Excepciones & Paso & Acción \\
\cline{2-3}
& 3 & Si la tarea solo tiene los campos minimos se puede agregar el cambio, así el caso de uso continua.   \\
\hline
Rendimiento & Paso & Cota de tiempo \\
\cline{2-3}
& 1 & 1 segundo \\
\cline{2-3}
& 2 & 1 segundo \\
\cline{2-3}
& 3 & 1 segundo \\
\cline{2-3}
& 4 & 5 segundos \\
\cline{2-3}
& 5 & 1 segundo \\
\hline
Importancia & \multicolumn {2}{p{10cm}|} {vital}    \\
\hline
Urgencia & \multicolumn {2}{p{10cm}|} {Hay presión}    \\
\hline
\end{tabular}
\end{table}

% 16.
\begin{table}[htb]
\centering
\begin{tabular}{|l|c|p{8cm}|}
\hline
RF-16 & \multicolumn {2}{p{10cm}|} {Sugerir horarios para realizar tareas.}    \\
\hline
Descripcion & \multicolumn {2}{p{10cm}|} {Según los horarios disponibles, al momentos de adicionar una tarea se hace una sugerencia de horario para realizarla.}\\
\hline
Precondición & \multicolumn {2}{p{10cm}|} {Se debe haber agregado una tarea, y deben haber horarios disponibles para hacer la recomendación.}\\
\cline{2-3}
Secuencia & Paso & Acción \\
\cline{2-3}
& 1 & Al momento de agregar una tarea, se revisan horarios disponibles. \\
\cline{2-3}
& 2 & Se revisan las variables de la tarea (dificultad, fecha de entrega). \\
\cline{2-3}
& 3 & Se hace la recomendación. \\
\hline
Postcondición & \multicolumn {2}{p{10cm}|} {La recomendación se dará al usuario dandole la posibilidad de tomarla o dejarla.} \\
\hline
Excepciones & Paso & Acción \\
\cline{2-3}
& 1 & Puede pasar que no haya horarios disponibles para hacer la recomendación, así el caso de uso termina.  \\
\cline{2-3}
& 2 & Si no hay variables de tarea definidos, se pasa al siguiente paso. \\
\hline
Rendimiento & Paso & Cota de tiempo \\
\cline{2-3}
& 1 & 5 segundos \\
\cline{2-3}
& 2 & 5 segundos \\
\cline{2-3}
& 2 & 1 segundo \\
\hline
Importancia & \multicolumn {2}{p{10cm}|} {Normal}    \\
\hline
Urgencia & \multicolumn {2}{p{10cm}|} {Puede esperar}    \\
\hline
Comentarios & No. & Descripción \\
\cline{2-3}
& 1 & El caso de uso esta bastante ligado a otros casos de uso, como puede apreciarse en el diagrama, sin embargo su relevancia no es la misma como la de los casos a los que apoya. \\
\hline
\end{tabular}
\end{table}

% 19.
\begin{table}[htb]
\centering
\begin{tabular}{|l|c|p{8cm}|}
\hline
RF-19 & \multicolumn {2}{p{10cm}|} {Alertar de la próxima entrega de una tarea. }    \\
\hline
Descripcion & \multicolumn {2}{p{10cm}|} {Se pretende avisar con tiempo prudencial que el ciclo de una tarea esta por finalizar, lo cual significa que debe ser terminada para ser entregada.}\\
\hline
Precondición & \multicolumn {2}{p{10cm}|} {La existencia de la tarea.}\\
\cline{2-3}
Secuencia & Paso & Acción \\
\cline{2-3}
& 1 & Se revisa el tiempo para que la tarea deba ser terminada comparandolo con el prudencial. \\
\cline{2-3}
& 2 & Se realiza el aviso. \\
\hline
Postcondición & \multicolumn {2}{p{10cm}|} {El usuario será avisado sobre la proximidad de su tarea.} \\
\hline
Excepciones & Paso & Acción \\
\cline{2-3}
& 1 & Si la tarea no cuenta con tiempo de realización se considera indefinida, así el caso de uso termina.  \\
\hline
Rendimiento & Paso & Cota de tiempo \\
\cline{2-3}
& 1 & 1 segundo \\
\cline{2-3}
& 2 & 1 segundo \\
\hline
Importancia & \multicolumn {2}{p{10cm}|} {Importante}    \\
\hline
Urgencia & \multicolumn {2}{p{10cm}|} {Puede esperar}    \\

\hline
\end{tabular}
\end{table}



