\chapter{Caso de Estudio}
\section{Introducción}
Desde que el ser humano cuenta con raciocinio , ha buscado organizarse, desarrollar metodologías y nuevas tecnologías que faciliten su diario vivir. Ha habido un recorrido histórico en el cual las necesidades humanas de optimización de tiempo y recursos han ido en aumento, así mismo las soluciones a éstas. 
En los últimos años se ha podido apreciar una constante migración al uso de tecnologías de la información que permiten realizar a cabo tareas en todos los ámbitos de forma óptima. Uno de los actores que más se han visto inmersos en la revolución digital son los estudiantes, pero en su contexto universitario, hace falta desarrollar estrategias que le permitan mejorar la gestión de tiempo de sus actividades académicas; por lo cual se buscará una solución tecnológica que se adapte a las necesidades de los universitarios.
\section{Objetivo General}
Desarrollar un software que gestione actividades y tiempos de las asignaciones académicas a estudiantes universitarios, utilizando los modelos y metodologías de ingeniería de software para mejorar la productividad del universitario.

\section{Objetivos Específicos}
\begin{enumerate}
	\item Analizar el problema teniendo en cuenta la observación de las necesidades del estudiante, para así enfocarse en estos elementos primordiales a la hora de desarrollar el software.
	\item Presentar una solución a nivel de software a partir del previo análisis del problema para finalmente implementarlo. 
	
\end{enumerate}
\section{Descripción del problema}
La vida universitaria y académica suele ser difícil de manejar debido a la cantidad de trabajos que se deben entregar diariamente, a la prioridad que cada una es para el usuario y a la gestión de tiempo para poder realizarlos. Tareas, trabajos, talleres y grandes proyectos son algunas de las actividades que un estudiante realiza durante su semestre; además de que cada uno tiene complejidad y tiempo de realización diferentes estimados por el estudiante.
Una solución factible es la utilización de un software gestor de tareas orientado a la organización y  optimización de actividades académicas.

\section{Alcance}
Este software tendrá la capacidad de gestionar los horarios de los estudiantes, añadir recordatorios de trabajos próximos a presentar y ofrecer el servicio de organizar en horarios la realización de las tareas pendientes. Esto se llevará a cabo de acuerdo a la complejidad de la actividad a realizar, en la cual se tomará en cuenta el nivel de dificultad, si se puede desarrollar en diferentes etapas y la fecha de entrega. 

El estudiante estará en la capacidad de añadir actividades, determinar la complejidad de éstas y asignarles un horario de realización que puede ser repartido en varios bloques cuando la tarea requiere de mucho tiempo. Adicionalmente, las actividades podrán personalizarse añadiendoles objetivos a cumplir o subactividades.

