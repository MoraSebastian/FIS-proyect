\chapter{Conclusiones}

\begin{itemize}
\item Luego de haber pasado por las diferentes etapas del proceso de ingenieria de softaware, se empieza a formar la noción del porqué existen estas prácticas, cuestión que va de la mano teniendo en cuenta el contexto, en el sentido de que el problema que se abordo, contempla una mayor magnitud, y aún en caso contrario, se hubiese podido dar el mismo tratamiento.

\item Entrando en contexto con las prácticas en sí, el proceso que las reúne, conlleva la idea de la planificación, término que en principio, no se entiende la concepción que posee en el contexto del software, sin embargo, es una idea que toma forma a medida que se realizan las etapas. Tanto es así, que en el momento de ejecutar cada etapa, en varias ocasiones fue necesario volver a una etapa anterior, revisar, y adaptar, síntoma claro de falta de visión al momento de ejecutar el análisis. Sin embargo cabe destacar que fue algo necesario, e incluso esperado al ser la primera vez en ejecutar un proceso de ingeniería de software.

\item Al ser el software en cuestión enfocado hacia los estudiantes, la concepción del problema basado en las necesidades que tendría un estudiante para manejar sus tiempos, no fue complicada, más bien las ideas plasmadas se orientan bastante en las necesidades generalizadas. El abarcar necesidades más específicas, no fue tenido en cuenta, pues hubiese supuesto una mayor complejidad, que viendo los tiempos disponibles para implementar el proyecto, no sería factible. 
\end{itemize}


\newpage